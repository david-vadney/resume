%!TEX TS-program = xelatex
%!TEX encoding = UTF-8 Unicode
% Awesome CV LaTeX Template for CV/Resume
%
% Original author:
% Claud D. Park <posquit0.bj@gmail.com>
% http://www.posquit0.com
%
% Modifications by:
% David Vadney <dvadney@outlook.com>
%
% Template license:
% CC BY-SA 4.0 (https://creativecommons.org/licenses/by-sa/4.0/)
%

%-------------------------------------------------------------------------------
% CONFIGURATIONS
%-------------------------------------------------------------------------------

% A4 paper size by default, use 'letterpaper' for US letter
\documentclass[11pt, a4paper]{awesome-cv}



\usepackage{fontspec}
\usepackage{hyperref}
\usepackage{lastpage}
\usepackage{graphicx,calc}
\definecolor{link}{RGB}{28,74,238} 


\setmainfont{Times New Roman}

% Configure page margins with geometry
\geometry{left=1.4cm, top=.8cm, right=1.4cm, bottom=1.8cm, footskip=.5cm}

% Color for highlights
% Awesome Colors: awesome-emerald, awesome-skyblue, awesome-red, awesome-pink, awesome-orange, awesome-nephritis, awesome-concrete, awesome-darknight, awesome-SEUgreen
\colorlet{awesome}{awesome-darknight}
% Uncomment if you would like to specify your own color
%\definecolor{awesome}{HTML}{4D7C2C}

% Colors for text
% Uncomment if you would like to specify your own color
% \definecolor{darktext}{HTML}{414141}
% \definecolor{text}{HTML}{333333}
% \definecolor{graytext}{HTML}{5D5D5D}
% \definecolor{lighttext}{HTML}{999999}

% Set false if you don't want to highlight section with awesome color
\setbool{acvSectionColorHighlight}{true}

% If you would like to change the social information separator from a pipe (|) to something else
\renewcommand{\acvHeaderSocialSep}{\quad\textbar\quad}

\makeatletter
\patchcmd{\@sectioncolor}{\color}{\mdseries\color}{}{}
\makeatother

%-------------------------------------------------------------------------------
%	PERSONAL INFORMATION
%	Comment any of the lines below if they are not required
%-------------------------------------------------------------------------------
% Available options: circle|rectangle,edge/noedge,left/right
%\photo[rectangle,noedge,right]{picture.jpg}
\name{David}{Vadney}
\position{Software Engineer}
\email{dvadney@outlook.com}
%\birth{199x}
%\homepage{http://website.com/}
\github{https://github.com/david-vadney}
\linkedin{linkedin.com/in/dvadney}
%\gitlab{dvad924}
%\stackoverflow{SO-id}{SO-name}
%\twitter{@twit}
%\skype{skype-id}
%\reddit{reddit-id}
%\extrainfo{extra informations}
\location{Queens, NY}


%-------------------------------------------------------------------------------



\begin{document}

% Print the header with above personal informations
% Give optional argument to change alignment(C: center, L: left, R: right)
\makecvheader[C]

% Print the footer with 3 arguments(<left>, <center>, <right>)
% Leave any of these blank if they are not needed
\makecvfooter
{Last updated: \today}%{\today}
{David Vadney ~~~·~~~Résumé}%Résumé
{\thepage \ / \pageref{LastPage}}%{}

%-------------------------------------------------------------------------------
%	CV/RESUME CONTENT
%	Each section is imported separately, open each file in turn to modify content
% -------------------------------------------------------------------------------
\cvsection{Summary}

\begin {cvparagraph}

 Engineer with 9+ years of experience in analysis, development, and operations
 that is focused on learning and delivering reliable performant solutions to
 problems. Seeking a Senior Engineer role to help solve interesting problems.

\end{cvparagraph}

\vspace{-3mm}
\cvsection{Education}

\begin{cventries}
   \cventry
   {Master's in Computer Science}
   {University at Albany}
   {Albany, NY}
   {Sep. 2014 - May 2016}
   {}
   \vspace{-3.5mm}

\vspace{1.0mm}
   \cventry
   {Dual Bachelor's in Computer Science and Mathematics}
   {University at Albany}
   {Albany, NY}
   {Sep. 2010 - May 2014}
   {}
   \vspace{-3.5mm}

\end{cventries}
\vspace{-3mm}
\cvsection{Experience}

\begin{cventries}
  \vspace{1mm}
  \cventry
  {Tech Lead, Devops Platform}
  {Charter Communications (ThinkBRQ contractor)}
  {Denver, CO (Remote)}
  {Jan. 2021 - Present}
  {
    \begin{cvitems} % Description(s) of tasks/responsibilities
		\item {Lead a team of engineers in developing a library of CI/CD templates
        to reduce developer maintenance of pipelines, and greatly speed up
        operationalization of applications for an org of 10+ dev teams}
		\item {Helped the team create an API for short-term AWS creds for use in
        pipelines to prevent leaking credentials and promote principle of least
        privilege}
    \item {Assisted an org in migrating app telemetry data to Datadog and
        configuring k8s deployments}
    \item {Worked with infrastructure to setup Fluentd to allow log forwarding
        from k8s to splunk}
    \item {Gave many presentations to teams in the org about CI, SDLC, and secrets
        management for apps and pipelines}
    \end{cvitems}
  }

  \cventry
  {Software Engineer, SRE}
  {Cox Automotive (ThinkBRQ contractor)}
  {New Hyde Park, NY}
  {May 2018 - Dec. 2020}
  {
    \begin{cvitems}
        \item{ Improved agent start-up performance by 80\% during load spikes by
            processing the state of the job queue to help reduce pipeline duration}
        \item { Reduced application build times by nearly 50\% through
            concurrent layer builds and caching }
        \item { Added ECS Capacity Provider support to Jenkins plug-in to allow
            for safe and simple cluster scale-in }
        \item { Added support for DotNet builds and ephemeral windows container
            builds to speed up existing pipelines and added reliability for
            windows applications}
     \end{cvitems}
  }


  \cventry
  {Software Engineer/ Analyst}
  {AVAILabs}
  {Albany, NY}
  {Oct. 2014 - May. 2018}
  {
    \begin{cvitems}
    \item {Constructed an interactive mapping application to visualize the NPMRDS traffic and other datasets}
    \item {Built data processor to construct directed adjacency graphs from
        noisy point-data collections and metadata  }
    \item {Developed API that combined the graph associations with other datasets to allow for richer analysis}
    \end{cvitems}
  }
  \vspace{-3.5mm}

\end{cventries}
\cvsection{Technical Skills}
\vspace{-4.0mm}
\begin{cvskills}

  \cvskill
  {} % Type
  {Python, Django, Java, Maven, Gitlab, Terraform, Bash, Kubernetes, Docker, AWS, NewRelic,
    Datadog, Jenkins, Postgres, Oracle, MongoDB, Javascript/NodeJS , ReactJS,
    FluentD/FluentBit, and others... } % Skillset


  \vspace{-8.0mm}  
\end{cvskills}
%\input{resume/Research Projects.tex}
%\input{resume/Awards and Honors.tex}
%\input{resume/Volunteer Services.tex}
%\input{resume/References.tex}
%-------------------------------------------------------------------------------
\end{document}
%%% Local Variables:
%%% TeX-engine: xetex
%%% End:
